\documentclass[12pt,twoside]{article}
\usepackage{jmlda}
\usepackage{mathtext}
%\NOREVIEWERNOTES
\title
    [Тематический поиск схожих дел в коллекции актов арбитражных судов] % Краткое название; не нужно, если полное название влезает в~колонтитул
    {Тематический поиск схожих дел в коллекции актов арбитражных судов}
\author
    [Герасименко~Н.\,А.] % список авторов для колонтитула; не нужен, если основной список влезает в колонтитул
    {Герасименко~Н.\,А., Артёмова~Е.\,Л., Воронцов~К.\,В.} % основной список авторов, выводимый в оглавление
    [Герасименко~Н.\,А.$^1$, Артёмова~Е.\,Л.$^2$, Воронцов~К.\,В.$^3$] % список авторов, выводимый в заголовок; не нужен, если он не отличается от основного
\thanks
    {Задачу поставил:  Воронцов~К.\,В.
    Консультант:  Артёмова~Е.\,Л.}
\email
    {nikgerasimenko@gmail.com}
\organization
    {$^1$МАИ; $^2$НИУ ВШЭ; $^3$МФТИ}
\abstract
    {В работе рассматривается задача информационного поиска по коллекции актов арбитражных судов. В качестве запроса поисковой системе может выступать любой документ коллекции. В ответ на поисковый запрос генерируется список документов коллекции, ранжированный по убыванию релевантности. Для решения поставленной задачи построена тематическая модель коллекции актов арбитражных судов с помощью открытой библиотеки BigARTM. При построении модели  учтена специфика предметной области: добавлены модальность ссылок на нормативно-правовые акты, а также модальность юридических терминов, выделенных полуавтоматически, с использованием алгоритма TopMine.


\bigskip
\textbf{Ключевые слова}: \emph {LegalTech, BigARTM, тематическое моделирование, аддитивная регуляризация мультимодальных иерархических тематических моделей}.}

\begin{document}
\maketitle
%\linenumbers
\section{Введение}

%Целью данной работы является построение тематической модели коллекции актов арбитражных судов для осуществления разведочного поиска по ней.

%Разведочный поиск - это относительно новая парадигма информационного поиска, отличающаяся тем, что позволяет подавать на вход поисковой системе не короткий запрос, а целый документ или коллекцию документов. 

%Разведочный поиск - это относительно новая парадигма информационного поиска, предполагающая 

Специалисты в области юриспруденции регулярно сталкиваются в своей работе с необходимостью поиска документов в базах судебной практики. Зачастую они ищут дела, схожие с теми, над которыми работают, не всегда при этом зная, по какому запросу или в какой именно тематике искать. Данная задача может быть отнесена к классу задач информационного поиска.

Существует несколько парадигм информационного поиска, одной из которых является тематический разведочный поиск, в рамках которого в качестве запроса может выступать целый документ или даже коллекция документов.

%Данная задача может быть рассмотрена как задача тематического разведочного поиска и решить ее с помощью разработанных методов решения этой задачи.

Юридические тексты обладают особой спецификой и, не смотря на то, что могут иногда казаться обычными текстами на естественном языке, не могут в полной мере рассматриваться таким образом. Понимание истинного смысла юридического текста зачастую требует консультации с экспертом.

В ходе совместной работы с экспертами в области юриспруденции был выявлен ряд важных модальностей, без учета которых невозможно построение адекватной модели. Библиотека BigARTM\cite{BigARTM2015}, с помощью которой строится тематическая модель, обладает возможностями для учета этих модальностей. В данной работе учтена модальность ссылок на нормативно-правовые акты и модальность юридических терминов.

%Представляется перспективным применение системы тематического поиска для решения подобной задачи. 


%Целью данной работы является построение тематической модели коллекции актов арбитражных судов для осуществления разведочного поиска. 

%Такого рода поиск называется разведочным\cite{Ianina_2017}.

%, для его осуществления разработано множество методов. В данной работе сделана попытка применения одного из методов разведочного поиска, основанного на построении тематической модели, к задаче поиска по коллекции юридических документов.

%Для построения тематической модели используется библиотека BigARTM\cite{BigARTM2015}. Одним из ее преимуществ является возможность учета модальностей, с помощью которого сделана попытка учесть специфику предметной области, выделив ссылки на нормативно-правовые акты и юридические термины. Задача выделения именованных сущностей решается с помощью регулярных выражений и алгоритма TopMine для выделения коллокаций.

%модальности позволяют учесть дополнительную информацию при построении модели
%это очень важно, потому что область обладает большой специфичностью

Данное исследование основано на подходе, описанном в работе \cite{Ianina_2017}, где он использован при построении тематической модели для разведочного поиска по статьям на порталах Habrahabr.ru и TechCrunch.com. Построенная авторами модель показала хорошие результаты на размеченных с помощью ассесоров данных.

\section{Постановка задачи}

Дана коллекция, состоящая из 7937 юридических документов, а именно актов арбитражных судов относящихся к делам о банкротстве. Для каждого документа известно его место в иерархии категорий, например, категория <<Особенности банкротства отдельных категорий должников>> или подкатегория <<Внешнее урправление>>  категории <<Процедры банкротства>>.

Задача состоит в построении тематической модели данной коллекции с помощью библиотеки BigARTM. Перед построением тематической модели необходимо провести предварительную обработку текстов, которая заключается в следующих шагах.
\begin{enumerate} 
  \item Проанализировав структуру документов, требуется выделить из них, при помощи контекстно-свободных грамматик и регулярных выражений, ссылки на нормативно-правовые акты (НПА).
  \item Требуется выделить юридические темины с помощью алгоритма TopMine \cite{El-Kishky2014}.
\end{enumerate}
Ссылки на НПА и юридические термины будут учтены в модели в качестве модальностей.

%Предварительная оценка модели будет проводиться по критериям перплексии и разреженности, а окончательная - по методу, описанному в работе \cite{Ianina_2017}, а также на странице русскоязычного вики-ресурса MachineLearning.ru «Оценивание качества разведочного поиска (эксперимент)», основанная на асессорских оценках релевантности.

Предварительная оценка модели будет проводиться по критериям перплексии, разреженности распределений тем в документах, а также разреженности распределений токенов в темах для модальностей ссылок на НПА и юридических терминов в соответствии со стандартной методологией оценки \cite{vorontsov2015additive}. 

Окончательная оценка будет строиться следующим образом. Полученные в результате моделирования тематические вектора для каждого документа кластеризуются при помощи алгоритма k-means. Затем, по критерию Rand Index делается оценка, в какой степени картина кластеризации согласованна с картиной принадлежности документов их категориям:  оказались ли в одном кластере документы из одной категории, и оказались ли документы из разных категорий в разных кластерах.

Таким образом, формальная постановка задачи может быть сформулирована в следующем виде.
Пусть $D$ - коллекция документов, $|D|=n$. Каждый документ $d\in D$ принадлежит одной из категорий $T_{i}\in T$. С другой стороны, каждый документ $d$ принадлежит одному из кластеров $Y_{j}\in Y$, полученных в результате применения алгоритма k-means к множеству тематических векторов документов коллекции. Требуется построить тематическую модель таким образом, что


\begin{equation}
\frac{a+b}{a+b+c+d} = \frac{a+b}{{n \choose 2 }}\to max,
\end{equation}

где:

$a = |S^{a}|,$ где $S^{a} = \{ (d_{i}, d_{j}) \mid d_{i}, d_{j} \in Y_{k}, d_{i}, d_{j} \in T_{l}\},$

$b = |S^{b}|,$ где $S^{b} = \{ (d_{i}, d_{j}) \mid d_{i} \in Y_{k_{1}}, d_{j} \in Y_{k_{2}}, d_{i} \in T_{l_{1}}, d_{j} \in T_{l_{2}}\},$

$c = |S^{c}|,$ где $S^{c} = \{ (d_{i}, d_{j}) \mid d_{i}, d_{j} \in Y_{k}, d_{i} \in T_{l_{1}}, d_{j} \in T_{l_{2}}\},$

$d = |S^{d}|,$ где $S^{d} = \{ (d_{i}, d_{j}) \mid d_{i} \in Y_{k_{1}}, d_{j} \in Y_{k_{2}}, d_{i}, d_{j} \in T_{l}\},$

$1 \leq i,j \leq n, i \neq j, 1 \leq k, k_{1}, k_{2} \leq r, k_{1} \neq k_{2}, 1 \leq l, l_{1},l_{2} \leq s, l_{1} \neq l_{2}.$


\section{Вероятностное тематическое моделирование на основе аддитивной регуляризации}

Пусть $M$ - множество модальностей, каждой из которой соответствует набор токенов $W_{m}$, называемый словарем модальности. Пусть $W$ - множество токенов из всех словарей, соответствующих модальностям из $M$. Каждый документ коллекции $D$ с длинной ${n_{d}}$  представляет собой набор токенов $w_{1},...,w_{n_{d}}$ из множества $W$.

В соответствии с теорией аддитивной регуляризации тематических моделей для каждой модальности вводится критерий логарифма правдоподобия и с помощью EM-алгоритма максимизируется их взвешенная сумма. Также к сумме добавляются регуляризаторы - дополнительные критерии, необходимые поскольку в общем случае задача имеет бесконечно много решений.

\emph{Регуляризаторы сглаживания и разреживания} имеют одинаковый вид и отличаются только знаками коэффициентов $\alpha$ и $\beta$, для регуляризатора разреживания они отрицательны.
\begin{equation}
R(\Phi,\Theta)=\beta \sum_{m \in M}\sum_{t \in T}\sum_{w \in W^m} \beta_{w} ln \phi_{wt} + \alpha \sum_{d \in D}\sum_{t \in T}\alpha_{t} ln\theta_{td}\to max.
\end{equation} 
Регулятор сглаживания вводит в модель требование схожести распределений $\phi_{wt}$ с распределением $\beta_{w}$ и $\theta_{td}$ с распределением $\alpha_{t}$. Регуляризатор разреживания, в свою очередь, способствует появлению нулевых элементов в распределенях $\phi_{wt}$ и $\theta_{td}$, что позволяет находить более компактные представления документов.

\emph{Регуляризатор декоррелирования} вводит в модель требование различности тем путем минимизации ковариации между столбцами матрицы $\Phi$.
\begin{equation}
R(\Phi)=-\tau \sum_{t \in T}\sum_{s \in T\backslash t}\sum_{w \in W} \phi_{wt}\phi_{ws} \to max.
\end{equation} 
Также побочным эффектом работы регуляризатора декоррелирования является разреживание матрицы $\Phi$, поэтому в случае его применения, можно не применять регуляризатор разреживания для нее.

\section{Вычислительный эксперимент}
Эксперимент проводился на коллекции, состоящей из 7937 юридических документов, а именно актов арбитражных судов относящихся к делам о банкротстве. О каждом документе была известна его категория, 
эта информация в последствии использовалась для внешней оценки качества модели. При построении модели использовались, помимо модальности текста, модальность ссылок на НПА и модальность юридических терминов. 

В рамках предобработки документов была проведена лемматизация при помощи морфологического анализатора pymorphy2, были исключены 5\% наиболее высокочастотных слов, а также слова общей лексики из списка stop-words библиотеки nltk.

При построении тематической модели использовались регуляризаторы декоррелирования и сглаживания для матрицы $\Phi$ терминов в темах и регуляризатор разреживания для матрицы $\Theta$ тем в документах. Выбор параметров модели производился путем перебора значений по сетке с использованием набора критериев качества: перплексия, разреженность распределений токенов в темах, разреженность распределений тем в документах. Подбор весов модальностей также производился с помощью перебора по сетке.

\bibliographystyle{plain}
\bibliography{Gerasimenko2019Project50}


\end{document}
