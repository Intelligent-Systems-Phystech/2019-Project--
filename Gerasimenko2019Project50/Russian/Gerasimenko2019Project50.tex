\documentclass[12pt,twoside]{article}
\usepackage{jmlda}
%\NOREVIEWERNOTES
\title
    [Тематический поиск схожих дел в коллекции актов арбитражных судов] % Краткое название; не нужно, если полное название влезает в~колонтитул
    {Тематический поиск схожих дел в коллекции актов арбитражных судов}
\author
    [Герасименко~Н.\,А.] % список авторов для колонтитула; не нужен, если основной список влезает в колонтитул
    {Герасименко~Н.\,А., Артёмова~Е.\,Л., Воронцов~К.\,В.} % основной список авторов, выводимый в оглавление
    [Герасименко~Н.\,А.$^1$, Артёмова~Е.\,Л.$^2$, Воронцов~К.\,В.$^3$] % список авторов, выводимый в заголовок; не нужен, если он не отличается от основного
\thanks
    {Задачу поставил:  Воронцов~К.\,В.
    Консультант:  Артёмова~Е.\,Л.}
\email
    {nikgerasimenko@gmail.com}
\organization
    {$^1$МАИ; $^2$НИУ ВШЭ; $^3$МФТИ}
\abstract
    {В работе рассматривается задача информационного поиска по коллекции актов арбитражных судов. В качестве запроса поисковой системе может выступать любой документ коллекции. В ответ на поисковый запрос генерируется список документов коллекции, ранжированный по убыванию релевантности. Для решения поставленной задачи построена тематическая модель коллекции актов арбитражных судов с помощью открытой библиотеки BigARTM. При построении модели  учтена специфика предметной области: добавлены модальность ссылок на нормативно-правовые акты, а также модальность юридических терминов, выделенных полуавтоматически, с использованием алгоритма TopMine.


\bigskip
\textbf{Ключевые слова}: \emph {LegalTech, BigARTM, тематическое моделирование, аддитивная регуляризация мультимодальных иерархических тематических моделей}.}

\begin{document}
\maketitle
%\linenumbers
\section{Введение}

Целью данной работы является построение тематической модели коллекции актов арбитражных судов для осуществления разведочного поиска по ней.

Специалисты в области юриспруденции регулярно сталкиваются в своей работе с необходимостью поиска документов в базах судебной практики. Зачастую они ищут дела, схожие с теми, над которыми работают, не всегда при этом зная, по какому запросу или в какой именно тематике искать. Такого рода поиск называется разведочным\cite{Ianina_2017}, для его осуществления разработано множество методов. В данной работе сделана попытка применения одного из методов разведочного поиска, основанного на построении тематической модели, к задаче поиска по коллекции юридических документов.

Для построения тематической модели используется библиотека BigARTM\cite{BigARTM2015}. Одним из ее преимуществ является возможность использования механизма модальностей, с помощью которого сделана попытка учесть специфику предметной области, выделив ссылки на нормативно-правовые акты и юридические термины. Задача выделения именованных сущностей решается с помощью регулярных выражений и алгоритма TopMine для выделения коллокаций.

%модальности позволяют учесть дополнительную информацию при построении модели
%это очень важно, потому что область обладает большой специфичностью

Исследование основано на подходе, описанном в работе \cite{Ianina_2017}, где он использован при построении тематической модели для разведочного поиска по новостям технологий на порталах Habrahabr.ru и TechCrunch.com. Построенная авторами модель показала хорошие результаты на размеченных с помощью ассесоров данных. 
\newpage
\bibliographystyle{plain}
\bibliography{Gerasimenko2019Project50}


\end{document}
