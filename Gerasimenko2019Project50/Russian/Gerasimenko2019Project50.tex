\documentclass[12pt,twoside]{article}
\usepackage{jmlda}
%\NOREVIEWERNOTES
\title
    [Тематический поиск схожих дел в коллекции актов арбитражных судов] % Краткое название; не нужно, если полное название влезает в~колонтитул
    {Тематический поиск схожих дел в коллекции актов арбитражных судов}
\author
    [Герасименко~Н.\,А.] % список авторов для колонтитула; не нужен, если основной список влезает в колонтитул
    {Герасименко~Н.\,А., Артёмова~Е.\,Л., Воронцов~К.\,В.} % основной список авторов, выводимый в оглавление
    [Герасименко~Н.\,А.$^1$, Артёмова~Е.\,Л.$^2$, Воронцов~К.\,В.$^3$] % список авторов, выводимый в заголовок; не нужен, если он не отличается от основного
\thanks
    {Задачу поставил:  Воронцов~К.\,В.
    Консультант:  Артёмова~Е.\,Л.}
\email
    {nikgerasimenko@gmail.com}
\organization
    {$^1$МАИ; $^2$НИУ ВШЭ; $^3$МФТИ}
\abstract
    {В работе рассматривается задача информационного поиска по коллекции актов арбитражных судов. В качестве запроса поисковой системе может выступать любой документ коллекции. В ответ на поисковый запрос генерируется список документов коллекции, ранжированный по убыванию релевантности. Для решения поставленной задачи построена тематическая модель коллекции актов арбитражных судов с помощью открытой библиотеки BigARTM. При построении модели  учтена специфика предметной области: добавлены модальность ссылок на нормативно-правовые акты, а также модальность юридических терминов, выделенных полуавтоматически, с использованием алгоритма TopMine.


\bigskip
\textbf{Ключевые слова}: \emph {LegalTech, BigARTM, тематическое моделирование, аддитивная регуляризация мультимодальных иерархических тематических моделей}.}

\begin{document}
\maketitle
%\linenumbers
\section{Введение}

%Целью данной работы является построение тематической модели коллекции актов арбитражных судов для осуществления разведочного поиска по ней.

%Разведочный поиск - это относительно новая парадигма информационного поиска, отличающаяся тем, что позволяет подавать на вход поисковой системе не короткий запрос, а целый документ или коллекцию документов. 

%Разведочный поиск - это относительно новая парадигма информационного поиска, предполагающая 

Специалисты в области юриспруденции регулярно сталкиваются в своей работе с необходимостью поиска документов в базах судебной практики. Зачастую они ищут дела, схожие с теми, над которыми работают, не всегда при этом зная, по какому запросу или в какой именно тематике искать. Задача, которую решают специалисты называется задачей информационного поиска, которая может быть решена методами тематического разведочного поиска.

Тематический разведочный поиск - это парадигма поиска, в которой в качестве запроса может выступать целый документ или даже коллекция документов.

%Данная задача может быть рассмотрена как задача тематического разведочного поиска и решить ее с помощью разработанных методов решения этой задачи.

Юридические тексты обладают особой спецификой и, не смотря на то, что могут иногда казаться обычными текстами на естественном языке, не могут в полной мере рассматриваться таким образом. Понимание истинного смысла юридического текста зачастую требует консультации с экспертом.

В ходе совместной работы с экспертами в области юриспруденции был выявлен ряд важных модальностей, без учета которых невозможно построение адекватной модели. Библиотека BigARTM\cite{BigARTM2015}, с помощью которой строится тематическая модель, обладает возможностями для учета этих модальностей. В данной работе учтена модальность ссылок на нормативно-правовые акты и модальность юридических терминов.

%Представляется перспективным применение системы тематического поиска для решения подобной задачи. 


%Целью данной работы является построение тематической модели коллекции актов арбитражных судов для осуществления разведочного поиска. 

%Такого рода поиск называется разведочным\cite{Ianina_2017}.

%, для его осуществления разработано множество методов. В данной работе сделана попытка применения одного из методов разведочного поиска, основанного на построении тематической модели, к задаче поиска по коллекции юридических документов.

%Для построения тематической модели используется библиотека BigARTM\cite{BigARTM2015}. Одним из ее преимуществ является возможность учета модальностей, с помощью которого сделана попытка учесть специфику предметной области, выделив ссылки на нормативно-правовые акты и юридические термины. Задача выделения именованных сущностей решается с помощью регулярных выражений и алгоритма TopMine для выделения коллокаций.

%модальности позволяют учесть дополнительную информацию при построении модели
%это очень важно, потому что область обладает большой специфичностью

Данное исследование основано на подходе, описанном в работе \cite{Ianina_2017}, где он использован при построении тематической модели для разведочного поиска по статьям на порталах Habrahabr.ru и TechCrunch.com. Построенная авторами модель показала хорошие результаты на размеченных с помощью ассесоров данных.

\section{Постановка задачи}

Дана коллекция, состоящая из n (уточняется) юридических документов, актов арбитражных судов. Для каждого документа известна его тематика, которая не тождественна теме тематической модели, а носит более общий характер: тематика состоит из набора тем. 

Задача состоит в построении тематической модели данной коллекции с помощью библиотеки BigARTM. Проанализировав структуру документов, требуется выделить из них, при помощи контекстно-свободных грамматик и регулярных выражений, ссылки на нормативно-правовые акты (НПА). Также требуется выделить юридические темины с помощью алгоритма TopMine \cite{El-Kishky2014}. Ссылки на НПА и юридические термины будут учтены в модели как модальности.

Предварительная оценка модели будет проводиться по критериям перплексии и разреженности, а окончательная - по методу, описанному в работе \cite{Ianina_2017}, а также на странице русскоязычного вики-ресурса MachineLearning.ru «Оценивание качества разведочного поиска (эксперимент)», основанная на асессорских оценках релевантности.

\bibliographystyle{plain}
\bibliography{Gerasimenko2019Project50}


\end{document}
